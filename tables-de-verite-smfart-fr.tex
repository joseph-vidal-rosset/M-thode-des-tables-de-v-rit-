% Created 2020-03-22 dim. 11:55
% Intended LaTeX compiler: pdflatex
\documentclass[article,reqno,times,12pt,french]{smfart}
\usepackage[square,numbers]{natbib}
                \usepackage[french]{babel}
                \usepackage{smfenum}
                 \usepackage{smfthm}
                  \usepackage[square,numbers]{natbib}
\usepackage[utf8]{inputenc}
\usepackage{siunitx}
\usepackage{lmodern}
\usepackage[T1]{fontenc}
\usepackage{graphicx}
%\usepackage{etex}
\usepackage{xcolor}
\usepackage[normalem]{ulem}
\usepackage{textcomp}
\usepackage{marvosym}
\usepackage{wasysym}
\usepackage{amssymb}
\usepackage{amsmath}
\usepackage{smfthm}
\usepackage{qtree}
\usepackage{bussproofs}
\usepackage{proof}
\usepackage{cancel}
\usepackage{url}
\usepackage{pdflscape}
\usepackage{fontawesome}
\usepackage{tikz}
\usepackage[linktocpage,pdfstartview=FitH,colorlinks,linkcolor=blue,anchorcolor=blue,citecolor=blue,filecolor=blue,menucolor=blue,urlcolor=blue]{hyperref}
 \usepackage{hyperref}
\hypersetup{colorlinks=true, breaklinks=true, citecolor=blue, linkcolor=blue, menucolor=blue, urlcolor=blue} 
               \author{Joseph Vidal-Rosset}
	       \address{Universit\'e de Lorraine, D\'epartement de philosophie,} 
                \address{Archives Poincar\'e, UMR 7117 du CNRS,}
                 \address{ 91 bd Lib\'eration, 54000 Nancy, France.}
               \email{joseph.vidal-rosset@univ-lorraine.fr}

%\usepackage{scripttab}
\date{\today}
\title{La méthode des tables de vérité}
\begin{document}

\maketitle
\tableofcontents


\section{Syntaxe de la logique propositionnelle classique}
\label{sec:org2c4fe97}
\subsection{Connecteurs propositionnels}
\label{sec:org765e617}
\begin{itemize}
\item La \emph{négation},  la \emph{conjonction},  la \emph{disjonction},  le \emph{conditionnel}  et le
\emph{biconditionnel} sont les connecteurs usuels  du calcul propositionnel. On les
représente respectivement par les symboles suivants:
\end{itemize}
\[
\lnot, \land, \lor, \to, \leftrightarrow
\]
Ces symboles  sont appelés \og symboles de  connecteurs propositionnels\fg{},
parce qu'ils  sont les liens entre  formules \og atomiques\fg{} (c'est-à-dire
les variables propositionnelles) et  permettent ainsi la formation des
formules \og moléculaires\fg{},  dont parmi celles-ci, les  \emph{théorèmes} de la
logique propositionnelle classique. 
\subsection{Variables propositionnelles}
\label{sec:orge4e8216}
Les variables propositionnelles  que l'on note par les formules
      \[
       A, B, C, D, \dots
      \]
symbolisent   n'importe  quel   énoncé   que   l'on  considère   comme
\emph{indivisible} et, par conséquent, dénué de connecteur propositionnel.   
\begin{exem}
\og La neige  est blanche\fg{} est  un énoncé  du langage ordinaire  que l'on
peut considérer comme atomique car il ne comporte aucun terme que l'on
peut   traduire   formellement   par    un   symbole   de   connecteur
propositionnel. 
\end{exem}
\subsection{Constantes du Vrai et du Faux}
\label{sec:orgbaa765a}
Il  existe  des  énoncés  qui sont  logiquement  \emph{toujours  vrais}  et
d'autres qui  sont toujours  faux.  La  négation d'un  énoncé toujours
vrai est un énoncé toujours  faux et, réciproquement, la négation d'un
énoncé toujours faux est un énoncé toujours vrai.

\begin{exem}
\og Si Socrate est un philosophe, alors Socrate est un philosophe\fg{} est un
énoncé toujours  vrai \emph{en raison de sa forme}: car il reste vrai si  l'on remplace
\og Socrate est un  philosophe\fg{} par un énoncé faux ou  même par un énoncé
dont  on   ignore  la  signification   ou  encore  par   une  variable
propositionnelle: \og Si \#, alors \#\fg{} et \og \(A \to A\)\fg{} sont formellement ou
logiquement toujours vrais.
\end{exem}

\begin{exem}
\og L'eau est  un liquide et  l'eau n'est pas  un liquide\fg{} est  un énoncé
logiquement contradictoire, comme toute  conjonction d'un énoncé et de
la  négation  de  ce  même  énoncé:   \(\lnot  A  \land  A\)  est  une
contradiction et à ce titre est toujours faux. 
\end{exem}

On symbolise par 
\[
\top
\] 
un  énoncé qui  est  toujours  faux et  l'on  appelle  ce symbole  \og la
constante du vrai\fg{}.

On symbolise par 
\[
\bot
\]
un  énoncé qui  est  toujours  faux et  l'on  appelle  ce symbole  \og la
constante du faux\fg{}.

\begin{rema}
On  peut remplacer  \(\top\) par  n'importe quelle  formule \emph{F} qui  peut
faire l'objet  d'une preuve en  logique classique qui montre  que \emph{F}
est  vraie.  De même, on  peut  remplacer  \(\bot\) par  n'importe  quelle
formule \emph{G}  qui peut faire l'objet  d'une preuve qui montre  que \emph{G}
est une contradiction logique. 
\end{rema}

\subsection{Expressions bien formée ou \og formules\fg{}}
\label{sec:orgd85867a}
\begin{defi}
On  appelle \og formule\fg{}  une  expression bien  formée, c'est-à-dire  une
expression en conformité avec la syntaxe du calcul propositionnel. 
\end{defi}
\begin{defi}
[Formule].

\begin{enumerate}
\item Une  variable propositionnelle est une formule.
\item Si \emph{A} est une formule, alors \(\lnot A\) l'est aussi.
\item Si \emph{A}  et si \emph{B} sont  des formules, alors \[(A \land  B), (A \lor
   B), (A \to B), (A \leftrightarrow B)\] sont aussi des formules.
\item Rien d'autre n'est une formule.
\end{enumerate}
\end{defi}
\section{Sémantique du calcul propositionnel classique}
\label{sec:orgae6fb7a}
\begin{defi}
[Interprétation des formules] L'interprétation classique d'une formule
\emph{F} de la logique propositionnelle se  fait à partir de fonctions (les
valuations) qui associent, à  chaque variable propositionnelle de \emph{F},
l'une des deux valeurs de vérité de ce calcul: le \emph{vrai} ou le \emph{faux}.  
\end{defi}
\begin{defi}
[Satisfiabilité]  On  dit qu'une  formule  \emph{F}  est \og satisfiable\fg{}  (ou
\og cohérente\fg{} ou encore \og consistante\fg{}) si  et seulement si \emph{F} est vraie
pour au moins une de ses inteprétations \emph{I}, ce qui se note:
\begin{equation}
\label{eq:1}
I \models F
\end{equation}
\end{defi}
\begin{defi}
[Validité] 
On dit qu'une  formule est \emph{valide} ou \emph{tautologique}  si et seulement
si elle est vraie pour \emph{toutes} ses interprétations, ce qui s'écrit: 
\begin{equation}
\label{eq:2}
\emptyset \models F
\end{equation}
\begin{equation}
\label{eq:3}
\top \models F
\end{equation}
ou encore tout simplement
\begin{equation}
\label{eq:4}
\models F
\end{equation}
\end{defi}
\begin{defi}
[Insatisfiabilité] 
On  dit  qu'une formule  est  \og insatifiable\fg{}  (ou \og contradictoire\fg{}  ou
\og incohérente\fg{})  si et  seulement si  elle est  fausse pour  toutes ses
interprétations, ce qui se symbolise de la façon suivante:
\begin{equation}
\label{eq:5}
F \models \bot
\end{equation}
ou encore 
\begin{equation}
\label{eq:6}
F \models 
\end{equation}
puisque   la  contradiction   n'a  par   définition  \emph{aucun   modèle},
ce qui signifie qu'aucune inteprétation n'est susceptible de la rendre
vraie.  D'autre  part,  seule  une  formule  qui  est  contradictoire,
c'est-à-dire toujours fausse peut impliquer  ce qui est toujours faux.
La formule  \eqref{eq:5} peut aussi s'exprimer  par l'implication \(\bot
\Rightarrow \bot\) qui est  évidemment valide c'est-à-dire équivalente
à \(\top\).  
\end{defi}
\begin{rema}
\begin{itemize}
\item Si l'on  prouve que \emph{F}  est une  formule \emph{valide}, alors  on prouve
\eqref{eq:2}, ainsi  que \eqref{eq:3} et  \eqref{eq:4} qui disent  la même
chose.
\item Si  l'on prouve  que  \emph{F}  est une  contradiction,  alors on  prouve
\eqref{eq:5} ainsi que \eqref{eq:6} qui dit la même chose.
\end{itemize}
\end{rema}
\section{L'algorithme des tables de vérité}
\label{sec:org28fc6a1}
L'algorithme des tables de vérité est la méthode sémantique de base du
calcul propositionnel classique. Celle-ci est  fondée sur le fait que,
dans une  théorie qui  admet le  \emph{principe de  bivalence}, c'est-à-dire
l'idée que  tout énoncé est vrai  ou faux indépendamment de  la preuve
que l'on peut en donner, alors  si l'on formalise un énoncé quelconque
dans une  formule \emph{F} du  calcul propositionnel, \emph{F}  recoit \(2^{n}\)
interprétations  ou  valuations,  \emph{n}  étant le  nombre  de  variables
propositionnelles dans \emph{F}, le nombre d'\emph{atomes}  dans \emph{F}. Si \emph{n = 1},
c'est-à-dire si \emph{F} est atomique,  alors le nombre des interprétations
possibles de \emph{F}  est \(2^{1}\), c'est-à-dire 2 (le vrai  et le faux).
S'il \emph{F}  a deux atomes,  le nombre d'inteprétations possibles  de \emph{F}
est de \(2^{2}\) = 4, trois atomes \(2^{3}\) = 8 inteprétations, etc. 
\subsection{Table de la négation}
\label{sec:org96ceec5}
\begin{center}
\begin{tabular}{cc}
\(A\) & \(\lnot A\)\\
\(\top\) & \(\bot\)\\
\(\bot\) & \(\top\)\\
\end{tabular}
\end{center}
\subsection{Table de la conjonction}
\label{sec:orga20bbbe}
\begin{center}
\begin{tabular}{ccc}
\(A\) & \(B\) & \(A \land B\)\\
\(\top\) & \(\top\) & \(\top\)\\
\(\top\) & \(\bot\) & \(\bot\)\\
\(\bot\) & \(\top\) & \(\bot\)\\
\(\bot\) & \(\bot\) & \(\bot\)\\
\end{tabular}
\end{center}
\subsection{Table de la disjonction}
\label{sec:orgb95d135}
\begin{center}
\begin{tabular}{ccc}
\(A\) & \(B\) & \(A \lor B\)\\
\(\top\) & \(\top\) & \(\top\)\\
\(\top\) & \(\bot\) & \(\top\)\\
\(\bot\) & \(\top\) & \(\top\)\\
\(\bot\) & \(\bot\) & \(\bot\)\\
\end{tabular}
\end{center}
\subsection{Table du conditionnel}
\label{sec:orgb5395ef}
\begin{center}
\begin{tabular}{ccc}
\(A\) & \(B\) & \(A \to B\)\\
\(\top\) & \(\top\) & \(\top\)\\
\(\top\) & \(\bot\) & \(\bot\)\\
\(\bot\) & \(\top\) & \(\top\)\\
\(\bot\) & \(\bot\) & \(\top\)\\
\end{tabular}
\end{center}
\subsection{Table du biconditionnel}
\label{sec:orga75602b}
\begin{center}
\begin{tabular}{ccc}
\(A\) & \(B\) & \(A \leftrightarrow B\)\\
\(\top\) & \(\top\) & \(\top\)\\
\(\top\) & \(\bot\) & \(\bot\)\\
\(\bot\) & \(\top\) & \(\bot\)\\
\(\bot\) & \(\bot\) & \(\top\)\\
\end{tabular}
\end{center}

\begin{exem}
On teste la formule suivante
\begin{equation}
\label{eq:7}
((A \lor B) \to C) \leftrightarrow ((A \to C) \land (B \to C))
\end{equation}
avec le constructeur  automatique de tables de vérité qui  se trouve à
cette adresse: 
\url{https://mrieppel.net/prog/truthtable.html} 
Dans la syntaxe de ce prouveur \eqref{eq:7} se traduit par:

\begin{verbatim}
((A v B) > C) <> ((A > C) & (B > C))
\end{verbatim}
\end{exem}

%NOTE: requires \usepackage{color}
\begin{tabular}{@{ }c@{ }@{ }c@{ }@{ }c | c@{ }@{}c@{}@{}c@{}@{ }c@{ }@{ }c@{ }@{ }c@{ }@{}c@{}@{ }c@{ }@{ }c@{ }@{}c@{}@{ }c@{ }@{}c@{}@{}c@{}@{ }c@{ }@{ }c@{ }@{ }c@{ }@{}c@{}@{ }c@{ }@{}c@{}@{ }c@{ }@{ }c@{ }@{ }c@{ }@{}c@{}@{}c@{}@{ }c}
A & B & C &  & ( & ( & A & $\lor$ & B & ) & $\rightarrow$ & C & ) & $\leftrightarrow$ & ( & ( & A & $\rightarrow$ & C & ) & $\land$ & ( & B & $\rightarrow$ & C & ) & ) & \\
\hline 
$\top$ & $\top$ & $\top$ &  &  &  & $\top$ & $\top$ & $\top$ &  & $\top$ & $\top$ &  & \textcolor{red}{$\top$} &  &  & $\top$ & $\top$ & $\top$ &  & $\top$ &  & $\top$ & $\top$ & $\top$ &  &  & \\
$\top$ & $\top$ & $\bot$ &  &  &  & $\top$ & $\top$ & $\top$ &  & $\bot$ & $\bot$ &  & \textcolor{red}{$\top$} &  &  & $\top$ & $\bot$ & $\bot$ &  & $\bot$ &  & $\top$ & $\bot$ & $\bot$ &  &  & \\
$\top$ & $\bot$ & $\top$ &  &  &  & $\top$ & $\top$ & $\bot$ &  & $\top$ & $\top$ &  & \textcolor{red}{$\top$} &  &  & $\top$ & $\top$ & $\top$ &  & $\top$ &  & $\bot$ & $\top$ & $\top$ &  &  & \\
$\top$ & $\bot$ & $\bot$ &  &  &  & $\top$ & $\top$ & $\bot$ &  & $\bot$ & $\bot$ &  & \textcolor{red}{$\top$} &  &  & $\top$ & $\bot$ & $\bot$ &  & $\bot$ &  & $\bot$ & $\top$ & $\bot$ &  &  & \\
$\bot$ & $\top$ & $\top$ &  &  &  & $\bot$ & $\top$ & $\top$ &  & $\top$ & $\top$ &  & \textcolor{red}{$\top$} &  &  & $\bot$ & $\top$ & $\top$ &  & $\top$ &  & $\top$ & $\top$ & $\top$ &  &  & \\
$\bot$ & $\top$ & $\bot$ &  &  &  & $\bot$ & $\top$ & $\top$ &  & $\bot$ & $\bot$ &  & \textcolor{red}{$\top$} &  &  & $\bot$ & $\top$ & $\bot$ &  & $\bot$ &  & $\top$ & $\bot$ & $\bot$ &  &  & \\
$\bot$ & $\bot$ & $\top$ &  &  &  & $\bot$ & $\bot$ & $\bot$ &  & $\top$ & $\top$ &  & \textcolor{red}{$\top$} &  &  & $\bot$ & $\top$ & $\top$ &  & $\top$ &  & $\bot$ & $\top$ & $\top$ &  &  & \\
$\bot$ & $\bot$ & $\bot$ &  &  &  & $\bot$ & $\bot$ & $\bot$ &  & $\top$ & $\bot$ &  & \textcolor{red}{$\top$} &  &  & $\bot$ & $\top$ & $\bot$ &  & $\top$ &  & $\bot$ & $\top$ & $\bot$ &  &  & \\
\end{tabular}
\end{document}